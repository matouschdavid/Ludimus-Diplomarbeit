\section{Vor dem Start}
Da die Ludimus-App ihre Verbindungen innerhalb des lokalen Netzwerks aufbaut, ist Voraussetzung, dass sich alle Geräte, auf denen der User Ludimus benutzten möchte, in einem lokalen Netzwerk befinden. Dies kann erreicht werden, indem man die Endgeräte alle mit dem gleichen W-Lan Netzwerk verbindet oder man sich einen Hotspot auf dem Smartphone erstellt und die Geräte damit verbindet. Zusätzlich muss auf allen Geräten, welche zum Spielen benutzt werden möchten, die Ludimus-App installiert sein. Dazu zählt auch der Laptop/Pc/SmartTv den man als Bildschirm benutzen möchte.
\section{Computer-App}
Wenn alle generellen Bedingungen erfüllt sind, kann man die Ludimus-App auf dem PC/Laptop starten. Nach einer kurzen Ladezeit wird dort den Usern ein Lobbycode und ein QR-Code angezeigt. Diese werden für die Verbindung mit dem Smartphone wichtig und da das Smartphone als Fernbedienung für die App fungiert kann der User sich nun zurücklehnen und bequem Ludimus bedienen.
\section{Android/IOS - App}
\subsection{Vor dem Einloggen}
Beim Starten der Ludimus-App öffnet sich zunächst die Login Maske, in welcher man gebeten wird sich einzuloggen. Wenn der User sich bereits einen Account erstellt hat, muss er nur seine Daten in der Maske ausfüllen und auf den Login Button drücken, um sofort erfolgreich eingeloggt zu werden. Wurde noch kein Account erstellt, hat der User zwei Möglichkeiten: Er kann sich einen neuen Account erstellen oder sich als Guest anmelden. Entscheidet er sich für die Variante Guest, hat er selber keine Spiele selbst zur Verfügung sondern kann nur Spielern mit bezahltem Account beitreten. In den Spielen ansich gibt es keinen Unterschied zwischen Guest und zahlendem Benutzer. Um sich einen Account zu erstellen, muss der auf den Sign-Up Button drücken. Es öffnet sich eine neue Maske, in welcher E-Mail und Passwort einzugeben sind. Aus Sicherheitsgründen wird die Eingabe des Passworts zweimal verlangt. Zum Erstellen des Accounts muss er die Eingabe mit Klick auf den nun rechts platzierten Sign-Up Button bestätigen. Der Account wird automatisch in der Firebase-Datenbank gespeichert und man wird sofort eingeloggt.
\subsection{Nach dem Einloggen}
Wenn der Spieler eingeloggt ist, kann er einen Namen und einen Lobbycode eingeben, dieser wird in der gestarteten Ludimus-App für den PC/Laptop angezeigt. Alternativ kann man auch den QR-Code, der den Lobbycode hinterlegt hat und ebenfalls in der Ludimus-App für PC/Laptop angezeigt wird, scannen und sich noch einfacher mit dem Server verbinden. Zum Scannen des QR-Codes klickt man auf den entsprechenden Button, es öffnet sich eine Kamera in der App und der QR-Code wird eingelesen. Sind beide Felder ausgefüllt hat, kann man sich mit einem Klick auf das W-Lan Symbol zu dem Server verbinden. Bei erfolgreicher Verbindung gibt es zwei Varianten. Möglichkeit 1: Der erste Spieler der sich zum Server verbindet ist der Admin der Lobby und kann Spiele starten und downloaden. Spiele werden per Klick auf das Symbol im Hauptmenü gestartet und damit auf allen Smartphones und auf dem PC/Laptop geladen. Möglichkeit 2: 
Man ist nicht der Admin der Lobby, sondern lediglich Mitspieler. In dieser Rolle wartet man, bis der Admin das Spiel startet, und dieses automatisch auf dem Smartphone lädt.
