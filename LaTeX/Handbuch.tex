\section{Vor dem Start}
Da die Ludimus-App ihre Verbindungen innerhalb des lokalen Netzwerks aufbaut, ist es vorausgesetzt das alle Geräte auf denen der User Ludimus benutzten möchte in einem lokalen Netzwerk hat. Dies kann erreicht werden in dem man die Endgeräte alle mit einem W-Lan Netzwerk verbindet oder indem man sich einen Hotspot auf dem Smartphone erstellt und die Geräte damit verbindet. Nun muss auf allen Geräten welche zum Spielen benutzt werden möchten die Ludimus-App installiert sein und schon kann man loslegen.
\section{Computer-App}
Wenn alle generellen Bedingungen erfüllt sind kann man die Ludimus-App auf dem PC/Laptop starten. Nach einer kurzen Ladezeit wird dem User ein Lobbycode und ein QR-Code angezeigt. Diese werden für die Verbindung mit dem Smartphone wichtig und da das Smartphone als Fernbedienung für die App fungiert kann der User sich nun zurücklehnen und bequem Ludimus bedienen.
\section{Android/IOS - App}
\subsection{Vor dem Einloggen}
Beim Starten der Ludimus-App öffnet sich zunächst die Login Maske in welcher man gebeten wird sich einzuloggen. Wenn der User sich bereits einen Account erstellt hat muss er nur seine Daten in der Maske ausfüllen und auf den Login Button drücken um sofort erfolgreich eingeloggt zu sein. Wenn man nun noch keinen Account hat, hat der User zwei Möglichkeiten, er kann sich einen neuen Account erstellen oder sich als Guest anmelden. Wenn er sich dafür entscheidet als Guest sich einzuloggen hat der User keine Spiele selbst zur Verfügung und kann nur Spielern mit bezahltem Account beitreten, in den Spielen selbst gibt es jedoch keinen Unterschied zwischen Guest und zahlenden Account. Wenn der User sich einen Account erstellen will muss er nur auf den Sign-Up Button drücken und es öffnet sich eine neue Maske in welcher der User seine E-Mail und sein Passwort, welches er zur Sicherheit zweimal angeben muss, eingeben kann. Zum Erstellen des Accounts muss er nun erneut auf den, nun rechts platzierten, Sign-Up Button drücken. Der Account wird automatisch in der Firebase-Datenbank gespeichert und man wird sofort eingeloggt.
\subsection{Nach dem Einloggen}
Wenn der Spieler nun eingeloggt ist kann er einen Name wählen und kann einen Lobbycode eingeben, dieser wird in der gestarteten Ludimus-App für den PC/Laptop angezeigt. Alternativ kann man jedoch auch den QR-Code, welcher ebenfalls den Lobbycode beinhaltet und welcher ebenfalls in der Ludimus-App für PC/Laptop angezeigt wird, scannen und somit mit noch weniger Arbeit sich zum Server verbinden. Zum Scannen des QR-Codes einfach auf den Button rechts unten drücken, der Button sieht aus wie ein QR-Code. Wenn man nun beide Felder ausgefüllt hat kann man auf das W-Lan Symbol drücken um sich zu dem Server zu verbinden. Bei erfolgreicher Verbindung gibts es nun zwei Möglichkeiten. Erstens, sie sind als erster Spieler verbunden. Sie sind nun der Admin dieser Lobby und können Spiele starten und downloaden. Zum Starten eines Spiels einfach auf das Symbol im Hauptmenü drücken und schon lädt das Spiel auf alle Smartphones und auf dem PC/Laptop. Wenn sie jedoch nicht der Admin, der Lobby sind können sie nun nur warten bis der Admin ein Spiel startet. Wenn er dies tut laden sich alle Dateien automatisch auf ihr Handy und sie können, nach einer kurzen Wartezeit, auch schon loslegen und spielen.