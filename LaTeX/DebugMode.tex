\section{Debug Modus} \label{sec:debug-modus}
Das in \ref{sec:spielstart} beschriebene System funktioniert einwandfrei mit fertigen Spielen. Da jedoch Spiele sowohl Datenbankeinträge, als auch hochgeladene Dateien am Fileserver benötigen, ist diese Lösung komplett ungeeignet für Spiele in Entwicklung. Der Debug Modus löst dieses Problem und ermöglicht es, auch kleine Änderungen, mit erheblich geringerem Zeitaufwand, zu testen.
Vor dem Debug Modus, wurden die Scripts der fertigen Spiele in das Hauptprojekt hinein kopiert. Die Entwicklung passierte in einem externen Projekt. Diese externen Projekte hatten ein Gerüst, das sich um die Spielerhandhabung und um die Kommunikation zwischen Client und Server im generellen kümmerte. Ein Testdurchlauf der aus „Assetbundles“ für Szenen und Modelle, jeweils wieder unterschieden in Client und Server, bauen, auf Fileserver hochladen, Scripts in Hauptprojekt kopieren und neuste Version für Smartphone und PC bauen dauerte mehrere Minuten.
Mit dem Debug Modus fällt das externe Projekt mit Kommunikationsgerüst, das Bauen der „Assetbundles“ und das hochladen auf den Fileserver weg. Für neue Spiele wird nur ein Ordner im Hauptprojekt und eine Client und Server Startszene angelegt. Die Kommunikation zwischen Server und Client ist mit Delegates abrufbar, wodurch man aktuelle Spieler, deren Inputs und vieles mehr gleich verwenden kann. Bestimmte Aspekte, wie Physics, UI oder Gegner-AI sind Client unabhängig und können somit nun, auch ohne diese, in der Engine, mit einem Knopfdruck, getestet werden. Werden sowohl Client, als auch Server benötigt, die Clients für Smartphone und PC bauen. Der letzte Schritt ist nur bei fast fertigen Versionen nötig, da es sonst eher ratsam ist, nur einen Client zu bauen und den Server in Engine zu debuggen.
Als erster verbundener Spieler kann man in der Spielanzeige das neue Spiel zwar nicht sehen, über die zwei Eingabefelder, kann man jedoch sein Spiel nun starten. Im linken Feld wird der Name der Startszene des Client angegeben und im rechten die des Servers. Drückt der Entwickler nun auf den Knopf in der Mitte wird sein Spiel gestartet. Die Zeiteinsparen und die Erleichterung der Entwicklung sind enorm, weshalb alle fertigen Spiele mit dieser Methode entwickelt worden sind.
