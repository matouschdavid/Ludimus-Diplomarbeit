\section{Sonstige Technologien}
\subsection{Figma} \label{sec:figma}
Figma ist ein Online Mockup Tool zum schnellen Erstellen von Userinterfaces. Entwürfe werden über die Cloud synchronisiert und können von mehreren Benutzern kollaborativ verwendet werden. Durch die Verfügbarkeit im Browser und das ermöglichen von Gruppenarbeiten, entschieden wir uns für Figma anstatt für Illustrator, welches wir noch zu Beginn benutzten.
\subsection{Blender} \label{Blender}
Blender ist ein Open Source 3D-Modellierungstool. Es ist gratis nutzbar und bietet neben der Modellerstellung noch viele weitere wichtige Features, wie Animationen, Materialerstellung oder die Kolorierung von Objekten. Durch die in Punkt \ref{sec:unity-blender} angesprochene Synergie, zwischen Blender und Unity, ist der Importprozess reibungslos.
\subsection{Vectr}
Vectr ist ein einfaches Tool zum Erstellen von Vector Grafiken. Die Funktionalität ist zwar auf ein paar Formen und Textfelder beschränkt, jedoch war dies ausreichend für all unsere Icons, weshalb wir von Photoshop zu Vectr wechselten.
\subsection{Illustrator}
Zu Beginn des Projektes benutzten wir Illustrator noch für die Mockup Erstellung. Wie in Punkt \ref{sec:figma} erwähnt, änderte sich dies jedoch und Illustrator wurde nur für die Erstellung des Billiardtisches, aufgrund der Mustererstellung für das Tuch, die Holzränder und die Metallecken, genutzt.
\subsection{Delegates (ES)}
\subsubsection{Übersicht}
Ein Delegat ist ein Typ, der Verweise auf Methoden mit einer bestimmten Parameterliste und dem Rückgabetyp darstellt. Nach Instanziierung eines Delegaten können Sie die Instanz mit einer beliebigen Methode verknüpfen, die eine kompatible Signatur und einen kompatiblen Rückgabetyp aufweist. Sie können die Methode über die Delegatinstanz aufrufen.
Delegaten werden verwendet, um Methoden als Argumente an anderen Methoden zu übergeben .Ereignishandler sind nichts weiter als Methoden, die durch Delegaten aufgerufen werden.
\subsubsection{Anwendung}
Um ein Delegate nutzen zu könnenen muss zuerst ein delegate erzeugt werden:
\newline
\tab public delegate string DoSth(string name, int y);
\newline
Beim erstellen des Delegates wird nach dem Keyword “delegate” der Rückgabewert der Methode bestimmt, in diesem Fall wird String zurückgegeben. Darauf folgt der Name, in unserem Fall “DoSth”, danach werden die Parameter welche die Methode benötigt noch angegeben und fertig ist das Delegate. Um dieses jedoch auch benutzen zu können muss es noch instanziiert werden. Dies kann man mit: public DoSth doSthHandler;. Nun kann das Delegate erst richtig benutzen. Nun kann dem Delegate eine Methode welche der obigen ovn der Signatur gleich hinzugefügt werden. Dies kann ganz einfach mit DoSth += myMethod; geschrieben werden. Wurde die Methode zum Delegate hinzugefügt kann man sie und alle anderen Methoden auf dem delegate mit: this.DoSth(“Parameter 1”, Parameter 2); aufrufen.
\subsubsection{Implementierung in Ludimus}
Bei Ludimus werden solche Delegates benutzt um Schnittstellen nach außen bereits zustellen. Diese Schnittstellen können benutzt werden um zum Beispiel eine Methode zum Player InputHandler hinzufügen und so aufgrund des gesendeten Key-Value Pair, des Smartphones, kann man selbst Befehle ausführen und zum Beispiel einen Charakter bewegen. Zusätzlich bietet die Nutzung von Delegates den großen Vorteil das man den Input auf jedes Spiel anpassen kann und somit Keywords, wie “move”, welche von mehreren Spielen verwendet werden öfters zu verwenden kann, da die Abfrage auf diese nicht von der Grundplattform durchgeführt wird sondern vom jeweiligen Spielscript.
\subsection{ZXing.Net (ES)}
\subsubsection{Übersicht}
Um einen QR-Code in C Sharp zu erzeugen wurde die ZXing.Net Library benutzt, welche die Entschlüsselung und Erzeugung von Barcodes unterstützt. ZXing.Net ist der Port der Java basierenden Library ZXing welche ebenfalls für Barcodes benutzt wird.
\newline \newline
ZXing.Net unterstütz eine Vielzahl von Barcode-Entschlüsselungen, um genau zu sein Unterstützt es: UPC-A, UPC-E, EAN-8, EAN-13, Code 39, Code 93, Code 128, ITF, Codabar, MSI, RSS-14 (all variants), QR Code, Data Matrix, Aztec and PDF-417.
\newline \newline
Der Barcode-Erzeuger unterstützt: UPC-A, EAN-8, EAN-13, Code 39, Code 128, ITF, Codabar, Plessey, MSI, QR Code, PDF-417, Aztec, Data Matrix
\subsubsection{Einbindung}
Grundsätzlich kann man die Library als NuGet Package downloaden. Dies ist jedoch in Unity nicht möglich. In Unity wird das ganze mithilfe einer .dll File gemacht welche in den Plugins-Folder in Unity gezogen wird. Schon kann man sie in den C Sharp Scripts benutzten.
\subsubsection{Warum ZXing.Net}
Die XZing.Net Library beitet eine Einbindung in Unity an und sie ist in C Sharp geschrieben. Sie ist  somit die einzige QR-Code Library welche es kostenfrei ermöglicht QR-Codes, in C Sharp zu erstellen und zu entschlüsseln. Einziger Negativpunkt ist die teils mangelnde Dokumentation.
\subsubsection{Usage}
\paragraph{Erzeugung}
In Unity muss man zur Benutzung der Library eine WebCamTexture instanzieren. Diese kann man mit “.Play()” und “.Stop()” starten und anhalten. Um nun aus der WebCamTexture Pixel auszulesen wird die Methode “.GetPixels32()” angewandt. Diese lieftert ein Color32 Array zurück. Mit diesem Array selbst fängt man wenig an, da es einfach nur ein Array von Farbwerten ist. Um mit dem Array arbeiten zu können muss zusätzlich noch einen IBarcodeReader instanziert werden. Dies geht mithilfe der “BarcodeReader()” Methode welche ZXing.Net beinhaltet und eben genau einen IBarcodeReader zurück gibt. Zum Entschlüsseln des Barcodes wird einfach die “.Decode(Color32 Array, WebCamTexture Höhe, WebCamTexture Breite)” Methode benutzt welche einen String, welcher den entschlüsselten QR-Code als Text beinhaltet, zurückliefert. Dieser wird das Color32 Array welches vorher erzeugt wurde als erster Parameter mitgegeben. Zusätzlich benötigt die Methode noch die Höhe und Breite der bereits instanzierten WebCamTexture. Dies geht dank den Properties der WebCamTexture “.height” und “.width”  ganz einfach. Um herauszufinden ob ein Barcode gefunden wurde wird einfach das Ergebnis der Decode Methode mit null verglichen, da es leer ist wenn kein Code gefunden wird.
\paragraph{Entschlüsselung}
Bei dem Erzeugen von Barcodes wird zunächst eine leere 2DTexture erzeugen, diese dient als an Anzeigefläche für den QR-Code. Darauf wird nun mit hilfe eines BarcodesWriters, welchem das Format “BarcodeFormat.QR\_CODE” zugewiesen wird und als Optionen die Höhe und Breite der 2DTexture mitgegeben werden. Mithilfe der “.Write(Text)” Methode bekommt man ein Color32 Array zurück, welches nun auf die 2DTexture geschrieben wird.
Nun wird diese noch im Frontend angezeigt und schon hat man einen QR-Code erzeugt.
\subsubsection{Implementierung in Ludimus}
Bei der Implementierung in Unity gab es ein gewaltiges Problem. Die Kamera war während der Anzeige im Frontend, wegen der Berechnung der ZXing.Net Library, ob ein Code gefunden wurde, nicht flüssig. Es wurden immer nur im Sekundenabstand Frames angezeigt. Auch wenn das reicht um den QR-Code zu scannen war es bei weitem nicht genug um auch benutzbar zu sein. Die Lösung war einfach einen Kamerastream zu zeigen mit welchem nichts berechnet wurde und welcher somit flüssig war.Dies jedoch klingt leichter als gedacht, denn in man kann in jeder Unity-Scene nur eine WebCamTexture am laufen haben. Als Endlösung wird in der Unity-Scene bereits eine WebCamTexture gestartet und dann die Berechnung des QR-Codes, mit einer neuen WebCam

