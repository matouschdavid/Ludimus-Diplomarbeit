\section{Sonstige Technologien}
\subsection{Figma} \label{sec:figma}
Figma ist ein Online Mockup Tool zum schnellen Erstellen von Userinterfaces. Entwürfe werden über die Cloud synchronisiert und können von mehreren Benutzern kollaborativ verwendet werden. Durch die Verfügbarkeit im Browser und das ermöglichen von Gruppenarbeiten, entschieden wir uns für Figma anstatt für Illustrator, welches wir noch zu Beginn benutzten.
\subsection{Blender}
Blender ist ein Open Source 3D-Modellierungstool. Es ist gratis nutzbar und bietet neben der Modellerstellung noch viele weitere wichtige Features, wie Animationen, Materialerstellung oder die Kolorierung von Objekten. Durch die in Punkt \ref{sec:unity-blender} angesprochene Synergie, zwischen Blender und Unity, ist der Importprozess reibungslos.
\subsection{Vectr}
Vectr ist ein einfaches Tool zum Erstellen von Vector Grafiken. Die Funktionalität ist zwar auf ein paar Formen und Textfelder beschränkt, jedoch war dies ausreichend für all unsere Icons, weshalb wir von Photoshop zu Vectr wechselten.
\subsection{Illustrator}
Zu Beginn des Projektes benutzten wir Illustrator noch für die Mockup Erstellung. Wie in Punkt \ref{sec:figma} erwähnt, änderte sich dies jedoch und Illustrator wurde nur für die Erstellung des Billiardtisches, aufgrund der Mustererstellung für das Tuch, die Holzränder und die Metallecken, genutzt.