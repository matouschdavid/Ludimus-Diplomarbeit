\section{Lobbycodes (ES)} \label{lobbycodes}
Das Verbinden zur der Laptop/PC Version von Ludimus erfolgt mit sogenannten Lobbycode. Dieser wird vom Computer aufgrund der IP-Adresse erzeugt und ermöglicht es einem Smartphone eine Verbindung zum Rechner aufzubauen.
\subsection{Verbindungsgeschichte}
Die Grundart der Verbindung stand von Anfang an fest: Network Streams. Aber wie genau soll sich der User nun, möglichst einfach, mit dem Gerät verbinden? Eine Möglichkeit dafür wäre ein Broadcast. Dabei wird ein Datenpaket an alle, mit dem gleichen Netzwerk verbundenen Geräte geschickt. Bei Ludimus konkret sollte jedes Gerät gefragt werden, ob es die Ludimus-App gestartet hat und sich dann bei erfolgreicher Rückmeldung zu der IP-Adresse des Antwortgeräts verbindet. Dieser Approach ist vermutlich der anwenderfreundlichste, da der User hier nichts tun muss und das Gerät sich beim Start der App automatisch zum PC/Laptop verbindet. 
Problem bei diesem Ansatz ist, da es nicht bei jedem Netzwerk mit nur einer Implementierung funktioniert, da Netzwerke teilweise sehr unterschiedlich konfiguriert sind. Manche Netzwerke lassen eine Umsetzung in C\# nicht zu. Zusätzlich fehlte uns hier das nötige Wissen, um alle Arten eines C\# Broadcasts zu implementieren. Keine Broadcasts und dennoch das ganze Netzwerk erreichen? Mit unseren Kenntnissen nicht lösbar. Folglich muss der User die IP-Adresse selbst eingeben, aber wie gestaltet man dies so userfreundlich wie nur möglich? Zum einen durch Kürzung der IP-Adresse und zum anderen durch einen QR-Code, welcher aufgrund der IP-Adresse erzeugt wird. So kann der User entweder einen kurzen Code eingeben oder seine Kamera benutzten zum Scannen des IP-QR-Codes.
\subsection{Erzeugung des Lobbycodes}
IP-Adressen sind nach dem Muster XXX.XXX.XXX.XXX aufgebaut, was bedeutet, der User müsste 12 Zahlen und 4 Punkte eingeben. Dies erschien uns zu unkomfortabel und dem User nicht zumutbar. Die Code-Eingabe kann man abkürzen, indem die Bytewerte der IP-Adresse Base64 verschlüsselt werden. Die Base64-Verschlüsselung erzeugt zwar ebenfalls etwas längere Codes, dennoch sind diese schon eindeutig einfacher als eine IP-Adresse. Zusätzliches Merkmal der Base64 Codierung ist, dass wenn die Länge des verschlüsselte Strings am Ende nicht auf ein Vielfaches von 3 kommt, der restliche Platz mit “=” aufgefüllt wird. Das bedeutet, der String “agFdeww” wird auf “agFdeww==” gestreckt. Bei der Codierung von Bytewerten der IP-Adressen trifft genau das zu. Man bekommt einen neunstelligen String, welcher am Schluss zwei “=” beinhaltet. Da dies immer so ist, kann man diese Zeichen für die Anzeige des Lobbycodes im Code wegschneiden und diese beim Eingeben des Codes am Smartphone einfach wieder hinzufügen. Somit enden wir bei einem siebenstelligen Code, der für den User einfach und schnell einzutippen ist. Zusätzlich verwandeln wir  diesen Code noch in einen QR-Code, der auf dem Smartphone gescannt werden kann, was die Codeeingabe für den User ebenfalls erleichtert und verkürzt. Im Code ist das ganze so umgesetzt:
\newline
\newline
\textbf{Erzeugung}
\newline  \newline
\textit{var b =BitConverter.GetBytes((int)IPAddress.Parse(this.ipAdress).Address);}
\newline
\textit{this.lobbycode = Convert.ToBase64String(b).Split('=')[0];}
\newline \newline
Zuerst wird mit einem BitConverter gearbeitet, um mit GetBytes() alle Bytewerte der IP-Adresse zu bekommen und folgend diese dann mit Base64 zu verschlüsseln. Mit this.lobbycode wird der Lobbycode einem Field übergeben und ist nun für alle Methoden verfügbar, zusätzlich werden hierbei bereits die “==” weggekürzt.
\newline \newline
\textbf{Entschlüsselung}
\newline \newline
\textit{var b = Convert.FromBase64String(code + "==");}
\newline
\textit{var a = new IPAddress(b);}
\newline
\textit{return a.ToString();}
\newline \newline
Hier wird der String, der den verschlüsselten Lobbycode beinhaltet, zunächst mit Base64 entschlüsselt, wobei hier wichtig ist, dass die zuvor gekürzten “==” wieder hinzugefügt werden. Anschließend werden mithilfe von new IPAddress(), die entschlüsselten Bytewerte zu einer IP-Adresse geparst und mit .ToString() wird ein String daraus.
\section{Verbindung (ES)} \label{verbindung}
Zur Verbindung der Smartphones mit dem Laptop/PC werden Sockets benutzt. Bei den Spielen und auch im Menü werden keine großen Mengen an Daten verschickt, da dies nicht nötig ist. Es genügt immer nur den jeweiligen Befehl per String zu schicken, und selbst wenn das öfters in der Sekunde passiert, reichen die Network Streams dafür perfekt aus und sind die beste Option. Dabei wird auf dem Laptop/PC, in einem Thread, eine Socket-Connection geöffnet welche wartet, bis sich jemand auf diese verbindet. Die Socket-Connection selbst wird mit dem Port 9393 geöffnet, das stellt sicher, dass sie von keiner anderen Anwendung benutzt wird und keine anderen Anwendung von ihr blockiert werden. Der Aufbau der Daten, welche per Network Stream versendet werden, sind Key-Value Pairs. Das bedeutet, jeder String, den wir verschicken, hat einen Key und ein Value, womit es ermöglicht wird, als Key einen Befehl zu schicken und als Value Parameter mit zu senden. Als Trennzeichen des Key-Value Pairs haben wir uns für “|” entschieden, da dies ein selten gebrauchtes Zeichen ist und am Smartphone nahezu unmöglich einzutippen ist. Zusätzlich wird der String mit einem “;” beendet. Es können mehrere Parameter mitgeschickt werden, welche dann mithilfe eines “:” von einander getrennt werden. Ein möglicher Informationssatz könnte demnach wie folgt aussehen: “Playername|Eric;” oder “Cards|K7:K3;”.
\section{Aufbau der Verbindung (ES)} 
\subsection{PC/Laptop/Tablet}
Bei der Desktop-Anwendung von Ludimus gibt es zunächst einen PlayerManager. Dieser regelt den Kommunikationsaustausch der Sockets, das Verbinden von neuen Smartphones oder das Verlassen bereits Verbundener. Der PlayerManager ist dabei ein Singelton, da seine Werte für alle immer gleich sein müssen. Wenn der PlayerManager das erste und einzige Mal instanziiert wird, öffnet er einen TCP-Listener in einem Thread. Dieser wartet so lange bis er geschlossen wird. Zusätzlich nimmt er keinen TCP-Client mehr an, wenn bereits acht Clients verbunden sind. Wenn sich ein Smartphone mit dem Rechner verbindet, wird automatisch ein Player Objekt erzeugt, welchem eine ID zugewiesen wird und welchem als Parameter der TCP-Client mitgegeben wird. Auch wird das Objekt in eine Liste aller Spieler hinzugefügt. Dieses Player Objekt bietet nun einige Grundfunktionen, wie zum Beispiel Nachrichten schicken. Es beinhaltet aber auch einen Thread, welcher die Bytes, die der Network Stream des TCP-Clients erhält, in einen String umwandelt und schlussendlich in die Unity Queue reiht. Die Unity Queue wird benötigt, um wichtige Befehle wie zum Beispiel einen Scenewechsel, durchzuführen, da solche Befehle nur im Main Thread erlaubt sind. Sie wird im Update von Unity öfters die Sekunde geprüft. Man kann mit “.enqueue(text)” Befehle einreihen indem man diese als Parameter übergibt. Mit “.Count” kann man überprüfen, ob in der Queue etwas eingereiht ist. Trifft dies zu kann man es mit “.dequeue()”, welches den Wert des eingereihten Strings zurückgibt, entreihen. Also reiht man den geparsten String, welcher vom Network Stream des TCP-Clients kommt, einfach im Thread ein und im Main Thread wird dann dieser String ausgewertet und aufgrund seines Wertes dementsprechend gehandelt. Standard sind dabei die Befehle:
\newline \tab 
- \textbf{GetID:} Die PlayerID wird dem TCP-Client geschickt.
\newline \tab  
- \textbf{RestartCurrentGame:} Startet die Scene in der sich der Spieler gerade befindet \tab neu.
\newline \tab 
- \textbf{CloseConnection:} Beendet die Verbindung mit dem TCP-Client sauber.
\newline \tab 
- \textbf{Playername:} Setzt den Spielernamen auf den auf “Playername” einen anderen \tab String.
\subsection{Android/IOS}
Bei Android/IOS ist das Ganze etwas einfacher, hier gibt es nur den ClientController, da hier nur ein TCP-Client, welcher sich auf die eingegebene IP-Adresse (mittels Lobbycode) verbindet. Es werden direkt beim Start zusätzlich benötige Infos angefragt und gesendet, damit das Playerhandling am PC/Laptop einwandfrei funktioniert. Auch hier wird die Unity Queue benutzt, um im Main Thread die wichtigen Befehle auszuführen. Standard sind hierbei die Befehle:
\newline \tab
- \textbf{CloseConnection:} Beenden der Anwendung und sauberes schließen der \newline \tab TCP-Client Verbindung.
\newline \tab
- \textbf{ID:} Property ID wird auf den mitgeschickten Value gesetzt.
\newline \tab 
- \textbf{CanRestart:} Führt alle Methoden, welche auf dem Restart Delegate hängen, \tab aus.
\section{Für Verbindung wichtige Klassen und Methoden / Übergang Schnittstellen (ES)} \label{fvwkum}
\subsection{Player}
\subsubsection{Übersicht}
Ein Player Objekt wird erzeugt, sobald sich ein neuer TCP-Client auf den TCP-Listener verbindet. Hat alle relevanten Infos, die der Spieler besitzt, wie zum Beispiel: Playername oder PlayerID. Beinhaltet einen Thread zur Kommunikation mit dem jeweiligen Smartphone.
\subsubsection{Wichtige Fields/Properties}
Name: \textbf{InputHandler}
\newline 
Info: Delegate auf welches man eine Methode der Signatur:
BeispielName(String key, String value) hängen kann. Wird aufgerufen wenn neue Daten per Network Stream eintreffen und es noch keinen vordefinierten Befehl für diese gibt.
\newline \newline
Name: \textbf{client}
\newline 
Info: TCP-Client, welcher auch den Network Stream beinhaltet. Verbindung zum Smartphone
\newline \newline
Name: \textbf{playerName}
\newline 
Info: Name des Spielers
\newline \newline
Name: \textbf{playerID}
\newline 
Info: ID des Spielers (reicht von 0-7, je nach Zeitpunkt des Verbindens)
\newline \newline
Name: \textbf{manager}
\newline
Info: Instanz des PlayerManager’s damit die Kommunikation zwischen Player und Koordinator reibungslos verläuft.
\newline \newline
\subsubsection{Wichtige Methoden}
Name: \textbf{sendData()}
\newline
Parameter: \textit{string key, string value}
\newline
Returnvalue: void
\newline
Info: Schickt dem TCP-Client per Network Stream ein Key-Value-Pair. Beim Senden mehrerer Values müssen diese bereits alle in dem Value Parameter vorhanden sein und mit einem “:” getrennt werden.
\newline \newline
Name: \textbf{kickPlayer()}
\newline
Parameter: \textit{keine}
\newline
Returnvalue: void
\newline
Info: Schließt sofort die Verbindung mit dem TCP-Client. Ermöglicht es, serverseitig Spieler zu entfernen.
\newline
\subsection{ClientController}
\subsubsection{Übersicht}
Dient zum Verbinden des Smartphones mit dem Laptop/PC. Läuft auf dem Smartphone. Startet einen TCP-Client, welcher sich zu der vorher per Lobbycode eingegebenen IP-Adresse verbindet. Hat ebenfalls einen Thread laufen, in welchem per Network Streams mit dem Server(Laptop/PC) kommuniziert wird. Ist ein Singelton.
\subsubsection{Wichtige Fields/Properties}
Name: \textbf{inputHandler}
\newline
Info: Delegate auf welches man eine Methode der Signatur: BeispielName(string key,string value) hängen kann. Wird aufgerufen, wenn neue Daten per Network Stream eintreffen und es noch keinen vordefinierten Befehl für sie gibt.
\newline \newline
Name: \textbf{restartHandler}
\newline
Info: Delegate auf welches man eine Methode der Signatur: BeispielName() hängen kann. Wird aufgerufen, wenn CanRestart vom Server geschickt wird.
\newline
\subsubsection{Wichtige Methoden}
Name: \textbf{getInstance()}
\newline
Parameter: \textit{keine}
\newline
Returnvalue: PlayerManager
\newline
Info: Singelton Konstruktor
\newline \newline
Name: \textbf{sendData()}
\newline
Parameter: \textit{string key, string value}
\newline
Returnvalue: void
\newline
Info: Schickt dem TCP-Client per Network Stream ein Key-Value-Pair. Beim Senden mehrerer Values müssen diese hier alle bereits in dem Value Parameter vorhanden sein und mit einem “:” getrennt werden.
\newline \newline
Name: \textbf{rename()}
\newline 
Parameter: \textit{string name}
\newline
Returnvalue: void
\newline
Info: Einfache Methode, welche es ermöglicht, den Player umzubenennen.
\newline 
