\section*{Declaration of Academic Honesty}
Hereby, I declare that I have composed the presented paper independently on my own and without any other resources than the ones indicated. All thoughts taken directly or indirectly from external sources are properly denoted as such.

This paper has neither been previously submitted to another authority nor has it been published yet. \\[1em]
Leonding, \duedateen \\[5em]
\ifthenelse{\isundefined{\firstauthor}}{}{\firstauthor}
\ifthenelse{\isundefined{\secondauthor}}{}{\kern-1ex, \secondauthor}
\ifthenelse{\isundefined{\thirdauthor}}{}{\kern-1ex, \thirdauthor}
\ifthenelse{\isundefined{\fourthauthor}}{}{\kern-1ex, \fourthauthor} \\[5em]

\begin{otherlanguage}{german}
\section*{Eidesstattliche Erklärung}
Hiermit erkläre ich an Eides statt, dass ich die vorgelegte Diplomarbeit selbstständig und ohne Benutzung anderer als der angegebenen Hilfsmittel angefertigt habe. Gedanken, die aus fremden Quellen direkt oder indirekt übernommen wurden, sind als solche gekennzeichnet.

Die Arbeit wurde bisher in gleicher oder ähnlicher Weise keiner anderen Prüfungsbehörde vorgelegt und auch noch nicht veröffentlicht. \\[1em]
Leonding, am \duedatede \\[5em]
\ifthenelse{\isundefined{\firstauthor}}{}{\firstauthor}
\ifthenelse{\isundefined{\secondauthor}}{}{\kern-1ex, \secondauthor}
\ifthenelse{\isundefined{\thirdauthor}}{}{\kern-1ex, \thirdauthor}
\ifthenelse{\isundefined{\fourthauthor}}{}{\kern-1ex, \fourthauthor} \\[5em]
\end{otherlanguage}

\begin{abstract}
	Ludimus is a mobile gaming platform for all sorts of card, board and arcade games designed for the modern family to play in their holidays. The goal is to provide an alternative to playing games with your family at home and to take all  your favourite games with you on your holidays. The latter heavily influenced our decisions on how we build our platform. Ludimus is not based on web technologies and remote servers because a good internet connection is rare in these occasions. When users have an internet connection they have the ability to download new games from our Server. We used Firebase for this, because we didn’t need any backend logic just storage and a database and that’s exactly what Firebase delivers on. We developed Ludimus with the game engine Unity, which provided the next to perfect cross platform capabilities we needed for our Windows, Android and iOS clients. We wanted to make playing in remote areas as easy as possible without ditching constructs that traditional games established. We think the reason why families play these games is because they involve local social activities, which get partially lost when playing over the internet and there is no central point of sight which players share. We tried to keep these basic principles but incorporated technology into the games as well. To accomplish this we decided to display the board, or the table in general, on a large screen, like a TV or a tablet and the cards, resources, controls or money of each player is displayed on their phone. We wanted to minimize the eye contact needed to control the phone, without abandoning it completely. Some people like the physical activity when rolling the dice or moving the figures on the board. That’s why we developed systems to simulate these gestures. 
	Although we attended many public events to show and promote Ludimus and took part in multiple competitions to win prices, we finally decided to cancel the project. 
	

\end{abstract}

\begin{otherlanguage}{german}
\begin{abstract}
	Ludimus ist eine mobile Spieleplattform für Karten-, Brett- und Arcadespielen, die für die moderne Familie in den Ferien konzipiert wurde. Das Ziel ist eine Alternative zum Spielen mit der Familie zu Hause anzubieten und zu ermöglichen alle Lieblingsspiele in den Urlaub mitzunehmen. Letzteres hat unsere Entscheidungen über den Aufbau unserer Plattform stark beeinflusst. Ludimus basiert nicht auf Webtechnologien und Remote-Servern, da im Urlaub eine gute Internetverbindung eine Seltenheit ist. Wenn Benutzer jedoch über eine Internetverbindung verfügen, können sie neue Spiele von unserem Server herunterladen. Hierfür haben wir Firebase verwendet, weil wir keine Backend-Logik brauchten, nur einen Dateispeicherort und eine Datenbank, und genau das ist, was Firebase liefert. Wir haben Ludimus mit der Game-Engine Unity entwickelt, da diese perfekte plattformübergreifende Funktionen bietet, die wir für unsere Windows-, Android- und iOS-Clients benötigten. Wir wollten das Spielen in abgelegenen Gegenden so einfach wie möglich machen, ohne Konstrukte zu verwerfen, die traditionelle Spiele etabliert haben. Wir denken, der Grund, warum Familien diese Spiele spielen, liegt darin, dass sie lokale soziale Aktivitäten beinhalten, die beim Spielen über das Internet teilweise verloren gehen und es keinen zentralen Punkt gibt, den die Spieler teilen. Wir haben versucht, diese Grundprinzipien beizubehalten, bauten jedoch auch Technologie in die Spiele ein. Um dies zu erreichen, haben wir uns entschieden, das Brett oder den Tisch im Allgemeinen auf einem großen Bildschirm wie einem Fernseher oder einem Tablet anzuzeigen, und die Karten, Ressourcen, Steuerelemente oder das Geld jedes Spielers werden auf dem Handy angezeigt. Wir wollten den Augenkontakt minimieren, der zur Steuerung des Smartphones erforderlich ist, ohne es vollständig unbrauchbar zu machen. Manche Leute mögen die körperliche Aktivität beim Würfeln oder beim Bewegen der Figuren auf dem Brett. Deshalb haben wir Systeme entwickelt, um diese Gesten zu simulieren.
	Obwohl wir bei vielen öffentlichen Veranstaltungen vertreten waren, um Ludimus zu zeigen und zu promoten, und an mehreren Wettbewerben teilgenommen haben, um Preise zu gewinnen, haben wir uns schließlich entschlossen, das Projekt abzubrechen.
	
	
	

\end{abstract}
\end{otherlanguage}

\section*{Danksagung}
Wir möchten uns bei der Htl Leonding und im speziellen bei Dipl.-Ing. Peter Bauer und Dipl.-Ing. Wolfgang Holzer für ihre Vision und ihr Engagement für die "Think Your Product"-Initiative bedanken. Bei unserem Betreuungslehrer Dipl.-Ing. Rupert Obermüller, der uns auf unserem Weg begleitet hat, möchten wir uns selbstverständlich auch bedanken, ebenso wie bei den Mitgliedern der Factory 300 und des Edison Preises für ihr Feedback und ihre Unterstützung.
