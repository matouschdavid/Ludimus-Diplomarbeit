\section*{Declaration of Academic Honesty}
Hereby, I declare that I have composed the presented paper independently on my own and without any other resources than the ones indicated. All thoughts taken directly or indirectly from external sources are properly denoted as such.

This paper has neither been previously submitted to another authority nor has it been published yet. \\[1em]
Leonding, \duedateen \\[5em]
\ifthenelse{\isundefined{\firstauthor}}{}{\firstauthor}
\ifthenelse{\isundefined{\secondauthor}}{}{\kern-1ex, \secondauthor}
\ifthenelse{\isundefined{\thirdauthor}}{}{\kern-1ex, \thirdauthor}
\ifthenelse{\isundefined{\fourthauthor}}{}{\kern-1ex, \fourthauthor} \\[5em]

\begin{otherlanguage}{german}
\section*{Eidesstattliche Erklärung}
Hiermit erkläre ich an Eides statt, dass ich die vorgelegte Diplomarbeit selbstständig und ohne Benutzung anderer als der angegebenen Hilfsmittel angefertigt habe. Gedanken, die aus fremden Quellen direkt oder indirekt übernommen wurden, sind als solche gekennzeichnet.

Die Arbeit wurde bisher in gleicher oder ähnlicher Weise keiner anderen Prüfungsbehörde vorgelegt und auch noch nicht veröffentlicht. \\[1em]
Leonding, am \duedatede \\[5em]
\ifthenelse{\isundefined{\firstauthor}}{}{\firstauthor}
\ifthenelse{\isundefined{\secondauthor}}{}{\kern-1ex, \secondauthor}
\ifthenelse{\isundefined{\thirdauthor}}{}{\kern-1ex, \thirdauthor}
\ifthenelse{\isundefined{\fourthauthor}}{}{\kern-1ex, \fourthauthor} \\[5em]
\end{otherlanguage}

\begin{abstract}
Here it is described what the thesis is all about. The abstract shall be brief and concise and its size  shall not go beyond one page. Furthermore it has no chapters, sections etc. Paragraphs can be used to structure the abstract. If necessary one can also use bullet point lists but care must be taken that also in this part of the text full sentences and a clearly readable structure are required.

Concerning the content the following points shall be covered. 

\begin{enumerate}
	\item {\em Definition of the project:} What do we currently know about the topic or on which results can the work be based? What is the goal of the project? Who can use the results of the project?
	
	\item {\em Implementation:} What are the tools and methods used to implement the project?
	
	\item {\em Results:} What is the final result of the project?
\end{enumerate}
This list does not mean that the abstract must strictly follow this structure. Rather it should be understood in that way that these points shall be described such that the reader is animated  to dig further into the thesis.

Finally it is required to add a representative image which describes your project best. The image here shows Leslie Lamport the inventor of \LaTeX.

\begin{flushright}
	\includegraphics[scale=.25]{images/leslie_lamport.jpg}
\end{flushright}

\end{abstract}

\begin{otherlanguage}{german}
\begin{abstract}
An dieser Stelle wird beschrieben, worum es in der Diplomarbeit geht. Die Zusammenfassung soll kurz und prägnant sein und den Umfang einer Seite nicht übersteigen. Weiters ist zu beachten, dass hier keine Kapitel oder Abschnitte zur Strukturierung verwendet werden. Die Verwendung von Absätzen ist zulässig. Wenn notwendig, können auch Aufzählungslisten verwendet werden. Dabei ist aber zu beachten, dass auch in der Zusammenfassung vollständige Sätze gefordert sind.

Bezüglich des Inhalts sollen folgende Punkte in der Zusammenfassung vorkommen: 

\begin{itemize}
	\item {\em Aufgabenstellung:} Von welchem Wissenstand kann man im Umfeld der Aufgabenstellung ausgehen? Was ist das Ziel des Projekts? Wer kann die Ergebnisse der Arbeit benutzen?
	
	\item {\em Umsetzung:} Welche fachtheoretischen oder -praktischen Methoden wurden bei der Umsetzung verwendet?
	
	\item {\em Ergebnisse:} Was ist das endgültige Ergebnis der Arbeit?
\end{itemize}
Diese Liste soll als Sammlung von inhaltlichen Punkten für die Zusammenfassung verstanden werden. Die konkrete Gliederung und Reihung der Punkte ist den Autoren überlassen. Zu beachten ist, dass der/die LeserIn beim Lesen dieses Teils Lust bekommt, diese Arbeit weiter zu lesen.

Abschließend soll die Zusammenfassung noch ein Foto zeigen, das das beschriebene Projekt am besten repräsentiert. Das folgende Bild zeigt Leslie Lamport, den Erfinder von \LaTeX.

\begin{flushright}
	\includegraphics[scale=.25]{images/leslie_lamport.jpg}
\end{flushright}

\end{abstract}
\end{otherlanguage}

\section*{Acknowledgments}
If you feel like saying thanks to your grandma and/or other relatives.
