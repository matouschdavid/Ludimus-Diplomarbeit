\chapter{Einleitung}
\section{Ausgangssituation}
Klassische Brettspiele benötigen viel Platz zum Verstauen. Platz, den eine Familie eventuell nicht einmal daheim hat, aber schon gar nicht wenn sie in die Ferien fahren. Brettspiele benötigen viel zu viel Platz im Koffer und man muss sich, wenn man schon welche mitnimmt, für ein paar wenige entscheiden. Und wenn dem Kind dann zufällig die Spiele gerade nicht passen hat man ein Problem. Bei Ludimus hat man die komplette Spielesammlung in der Cloud und kann sie immer und überall mitnehmen ohne viel Platz zu brauchen.
\newline 
Auch werden Kinder immer abgeneigter gegenüber klassischen Brettspielen und wollen lieber was am Handy spielen. Mithilfe von Ludimus können Eltern das Spielen mit ihren Kindern mit dem,  lieber am Handy spielen ihrer Kinder verbinden, da die Plattform komplett über das Smartphone gesteuert wird.

\section{Ziele}
Das große Ziel von Ludimus ist es, ein Produkt auf den Markt zu bringen, das die Art, wie Familien mit ihren Kindern Zeit verbringen, für immer ändert. Brettspielabende werden immer seltener, weil Kinder an ihre technischen Geräte heutzutage so sehr gewohnt sind, dass Brettspiele einfach ein zu großer Rückschritt sind. Ludimus soll einerseits diese Abende durch Technologie wieder regelmäßiger machen, andererseits aber auch Familien ermöglichen Spiele in den Urlaub mitzunehmen. Ludimus basiert auf mehreren Smartphones und einem Tablet, Laptop oder Fernseher. Mit Ausnahme des Fernsehers, sind das tragbare Geräte, die sowohl wenig Platz brauchen, als auch wenig Gewicht haben und somit selbst im Flieger mitgenommen werden können. 
Das Ziel ist somit im Urlaub für Familien unersetzbar zu sein um dadurch zu Hause eine Alternative gegenüber traditionellen Spielen zu sein.

\section{Überblick}

\begin{figure}
\begin{center}
	\includegraphics[scale=0.3]{images/ludimusLogo.png}
\end{center}
	\caption{Ludimus Logo}
\end{figure}

\section{Marke}
Ludimus war von Beginn an als Produkt für Kunden gedacht, weshalb wir viel Zeit in den Namen und das Logo steckten. Während der Prototypphase nannten wir das Projekt ConnecTable, da wir den Tisch mit Handys verbinden. Zusätzlich bedeutet das englische Wort „connectable“ aber auch verbindbar, wodurch das Prinzip hinter der Idee verdeutlicht wird. In der Aussprache würde die Schreibweise jedoch verloren gehen, weshalb wir uns für einen neuen Namen entschieden. Ludimus war das Produkt dieser neuen Überlegungen und ist Latein für „wir spielen“. Ludimus sollte eine Plattform für Familien sein und das „wir“ in den Mittelpunkt stellen. Zusätzlich klingt Ludimus melodisch und ist einfach auszusprechen. In Verbindung mit dem Namen entwarfen wir unser Logo. Es soll sowohl einen Spielwürfel, als auch eine Sprechblase darstellen, da Spiele, die Unterhaltungen und Zusammenhalt fördern, unser Ziel waren. 


\section{Ähnliche Arbeiten und Projekte}
Here a survey of other work in and around the area of the thesis is given. The reader shall see that the authors of the thesis know their field well and understand the developments there. Furthermore here is a good place to show what relevance the thesis in its field has.

\section{Struktur der Arbeit}
%dsflkjas flaksjfl asdfj as lfjldsajflaksdjf sa dfjlasdkfj sadlfjasdklf als dfj l dfsdfsdfn chapter~\ref{cha:used-technologies} (\nameref{cha:used-technologies}) on page~\pageref{cha:used-technologies} we describe the used technologies.
Finally the reader is given a brief description what (s)he can expect in the thesis. Each chapter is introduced with a paragraph roughly describing its content.