\chapter{Einleitung}
\section{Ausgangssituation}
Klassische Brettspiele benötigen viel Platz zum Verstauen. Platz den eine Familie eventuell nicht einmal daheim hat, aber schon gar nicht wenn sie in die Ferien fahren. Brettspiele benötigen viel zu viel Platz im Koffer und man muss sich, wenn man welche mitnimmt, für ein paar wenige entscheiden. Und wenn dem Kind dann zufällig die Spiele gerade nicht passen hat man ein Problem. Bei Ludimus hat man die komplette Spielesammlung in der Cloud und kann sie immer und überall mitnehmen ohne viel Platz zu verbrauchen.
\newline 
Auch werden Kinder immer abgeneigter gegenüber klassischen Brettspielen und wollen lieber am Handy spielen. Mithilfe von Ludimus können Eltern das Spielen mit ihren Kindern mit dem, lieber am Handy spielen ihrer Kinder verbinden, da die Plattform komplett über das Smartphone gesteuert wird.

\section{Ziele}
Das große Ziel von Ludimus ist es, ein Produkt auf den Markt zu bringen, das die Art, wie Familien mit ihren Kindern Zeit verbringen, für immer ändert. Brettspielabende werden immer seltener, weil Kinder an ihre technischen Geräte heutzutage so sehr gewohnt sind, dass Brettspiele einfach ein zu großer Rückschritt sind. Ludimus soll einerseits diese Abende durch Technologie wieder regelmäßiger machen, andererseits aber auch Familien ermöglichen Spiele in den Urlaub mitzunehmen. Ludimus basiert auf mehreren Smartphones und einem Tablet, Laptop oder Fernseher. Mit Ausnahme des Fernsehers, sind das tragbare Geräte, die sowohl wenig Platz brauchen, als auch wenig Gewicht haben und somit selbst im Flieger mitgenommen werden können. 
Das Ziel ist somit im Urlaub für Familien unersetzbar zu sein um dadurch zu Hause eine Alternative gegenüber traditionellen Spielen zu sein.

\begin{figure}
\begin{center}
	\includegraphics[scale=0.3]{images/ludimusLogo.png}
\end{center}
	\caption{Ludimus Logo}
\end{figure}

\section{Marke}
Ludimus war von Beginn an als Produkt für Kunden gedacht, weshalb wir viel Zeit in den Namen und das Logo steckten. Während der Prototypphase nannten wir das Projekt ConnecTable, da wir den Tisch mit Handys verbinden. Zusätzlich bedeutet das englische Wort „connectable“ aber auch verbindbar, wodurch das Prinzip hinter der Idee verdeutlicht wird. In der Aussprache würde die Schreibweise jedoch verloren gehen, weshalb wir uns für einen neuen Namen entschieden. Ludimus war das Produkt dieser neuen Überlegungen und ist Latein für „wir spielen“. Ludimus sollte eine Plattform für Familien sein und das „wir“ in den Mittelpunkt stellen. Zusätzlich klingt Ludimus melodisch und ist einfach auszusprechen. In Verbindung mit dem Namen entwarfen wir unser Logo. Es soll sowohl einen Spielwürfel, als auch eine Sprechblase darstellen, da Spiele, die Unterhaltungen und Zusammenhalt fördern, unser Ziel waren. 


\section{Ähnliche Arbeiten und Projekte}
Ludimus ist nicht die einzige Spieleplattform. Neben traditionellen Spielekonsolen, wie der Xbox oder der Playstation, betreten immer mehr Konkurrenten, wie Google, Amazon, Microsoft, uvm. mit Gamestreaming Plattformen den Markt für mobiles Spielen. Gegenüber diesen Alternativen grenzen wir uns klar mit der Auswahl unserer Spiele ab, denn wir bieten Spiele, die speziell für die Familie und das spielen mit Freunden entwickelt wurden. Ein weiterer Unterscheidungspunkt ist unser möglicher Einsatz im Urlaub, da wir keinen externen Server zum Spielen verwenden, sondern ein lokales Gerät zum Server machen.

\section{Struktur der Arbeit}
\subsection{Verwendete Technologien}
Hier werden alle von Ludimus verwendeten Technologien kurz erklärt und deren Einsatz im Projekt gezeigt. 
\subsection{How to use}
In wenigen Sätzen wird in diesem Kapitel beschrieben unter welchen Voraussetzungen eine Lobby zustande kommen kann, und welche Schritte der User durchführen muss um sich dorthin zu verbinden.
\subsection{Ausgewählte Aspekte der Systemerstellung}
Dieses Kapitel behandelt besonders wichtige oder besonders schwierige Aspekte während der Entwicklung von Ludimus. 
\subsection{Spiele}
In diesem Abschnitt werden alle begonnenen Spiele von Ludimus aufgezählt und erklärt, warum wir nicht jedes Spiel in unsere Spielebibliothek aufnahmen.
\subsection{Ars Electronica Festival}
Von unseren Vorabüberlegungen, Vorbereitungen und Testresultaten wird in diesem Kapitel beschrieben. 