\section{Übersicht}
Ars Electronica ist eine der weltgrößten Bühnen für Medienkunst, ein Festival für digitale Musik, eine Messe für Kreativität und Innovation und Spielwiese für die nächste Generation – Ars Electronica ist ein weltweit einzigartiges Festival für Kunst, Technologie und Gesellschaft.
\section{Ludimus Testlauf}
\subsection{Wie es dazu kam}
Dank des Einsatzes von DI Professor Peter Bauer durften Teams aus dem ThinkYourProduct-Projekt ihre Ideen und Projekte auf dem AEC-Festival präsentieren. Das ThinkYourProduct-Projekt ist für alle 4ten Klassen der HTL Leonding zugänglich. In dessen Rahmen findet man neue Ideen arbeitet diese aus. Aus genau so einer Idee ist auch Ludimus, unsere Diplomarbeit entstanden. Team Ludimus sah dieses Festival als einen Testlauf um zu sehen wie unsere Idee ankommt und ob die Kinder und Eltern begeistert sind.
\subsection{Idee}\label{aecerror}
Wir bauen eine gemütliche Wohnzimmerumgebung nach und lassen die Menschen auf einer Couch sitzen während sie Ludimus testen, so der Grundgedanke. Auch war ein großer Bildschirm notwendig, wie sollen die Leute sonst das Spielgeschehen aus einer gewissen Distanz gut sehen? Zusätzlich wurde noch ein lokales Netzwerk benötigt, welches ermöglichte, sich mit der Ludimus-App (am Smartphone installiert), zum Laptop, welcher die Ludimus-App ebenfalls gestartet hat, zu verbinden. Noch ein paar Deko-Elemente und schon ist der Ludimus-Stand für das AEC-Festival fertig. Fehlte nur noch ein Spiel, welches schnell und lustig war, aber trotzdem so anspruchsvoll, dass es für kurze Zeit fesselte. Die Lösung: ein Jump and Run, bei dem die Spieler das Ziel schneller als ihre Mitspieler erreichen sollen. Und zur Belohnung erhält der Erste im Ziel mehr Punkte und darf das Level, in welchem gespielt wird, verändern. Mit Veränderung ist gemeint, dass der Spieler ein Objekt in der Spielwelt platzieren darf, um das Level leichter oder schwerer zu machen. Und jener Spieler, der zuerst dreimal gewinnt, ist Sieger. Ein kurzes aber trotzdem anspruchsvolles Spiel, das Spaß macht. Dabei gab es jedoch ein Problem: Das Spiel hatte nichts mit dem Error-Thema des Festivals zu tun. Kurzerhand bauten wir absichtlich einen “Bug” in das Spiel ein, der die Bewegung des Spieler mit den meisten Punkten einfriert. Sein Charakter kann sich erst wieder bewegen wenn der Spieler sein Smartphone schüttelt. Und fertig war die Idee für das perfekte Spiel für das Festival, welches sogar zum Thema passte.
\subsection{Realität}
Zunächst benötigten wir eine Sitzgelegenheit. Zum Glück hatte eine Klasse unserer Schule eine Couch, welche wir uns ausborgen durften. Bezüglich Bildschirm mussten wir uns auch keine Sorgen machen, denn auch hier war uns die HTL Leonding eine riesige Hilfe, indem sie uns einen großen Bildschirm, der sonst für Präsentationen gedacht war bereitstellte. Dieser hatte zwar eine kleine Anzeigeverzögerung zwischen Laptop und Bildschirm war aber ansonsten perfekt. Als nächstes benötigen wir noch ein lokales Netzwerk, damit sich die Spieler verbinden können. Dafür wurde ein Hotspot auf dem Smartphone eines Teammitgliedes erstellt und schon konnte man spielen. Als Deko-Elemente wurden ein paar Sessel und Blumen hinzugestellt und der Ludimus-Stand war offiziell komplett. Fehlte nur noch das Spiel. Dieses wurde liebevoll Jumpy genannt und ist im Kapitel Jumpy (\ref{jumpy}) genauer ausgeführt.
\section{Ergebnis}
Das AEC-Festival war für Team Ludimus durchaus erfolgreich. Wir erhielten durchgehend positives Feedback und waren angenehm überrascht, dass es so gut läuft. Im Verlauf der vier Tage war der Stand nur sehr selten wirklich leer und wir erhielten viele Meinungen und konstruktive Kritik. Auch waren viele andere Programmierer auf dem AEC-Festival mit welchen wir uns austauschen und uns auf diese Weise weiterbilden konnten. Über den Verlauf des gesamten Festivals loggten sich über 150 Besucher mit ihrem Google-Account auf der Ludimus-Website ein, um die App downzuloaden. Dies war ein riesiger Erfolg für unser Projekt. Allerdings realisierten wir auch, wie hartnäckig Eltern eigentlich sein können, wenn es sich um Videospiele oder moderne Technologie handelt. Den Satz: “Ich kann sowas ja gar nicht, muss ich gar nicht erst probieren” hörten wir locker über 20 mal Tag. Selbst wenn wir erklärten, dass es auch andere Spiele, gibt wie zum Beispiel Billard, blieben viele bei ihrer Meinung, dass sie sowieso nicht mit dem Smartphone umgehen können und es lieber gar nicht probieren wollen. Dank dieses Feedbacks realisierten wir eine große Herausforderung, denn wir müssen die Eltern in Zukunft auch dazu bringen, mit den Kindern zu spielen. Schließlich ist das der Zweck von Ludimus. Als Lösung wurden genaue Anleitungen hinzugefügt und es wurde immer wieder betont, wie wichtig es sei, dass Eltern mit ihren Kindern gemeinsam spielen und Spaß haben.
\section{Fazit}
Die Präsentation von Ludimus auf dem AEC-Festival war für uns ein großer Erfolg. Wir konnten viel aus dem Feedback lernen und dank der Menschen auf dem AEC-Festival unsere App verbessern.
