\section{Übersicht}
Ars Electronica ist eine der weltgrößten Bühnen für Medienkunst, ein Festival für digitale Musik, eine Messe für Kreativität und Innovation und Spielwiese für die nächste Generation – Ars Electronica ist ein weltweit einzigartiges Festival für Kunst, Technologie und Gesellschaft.
\section{Ludimus Testlauf}
\subsection{Wie es dazu kam}
Dank des Einsatzes von DI. Professor Peter Bauer durften Teams aus dem ThinkYourProduct-Projekt ihre Ideen und Projekte auf dem AEC Festival präsentieren. Das ThinkYourProduct-Projekt ist für alle 4ten Klassen der HTL Leonding zugänglich. In dessen Rahmen überlegt man sich neue Ideen und Technologien aus und arbeitet diese aus. Aus genau so einer Idee ist auch Ludimus, unsere Diplomarbeit entstanden. Team Ludimus sah dieses Festival als einen Testlauf um zu sehen ob unsere Idee ankommt und ob die Kinder oder Eltern begeistert sind.
\subsection{Idee}\label{aecerror}
Wir bauen eine gemütliche Wohnzimmerumgebung nach und lassen die Menschen auf einer Couch sitzen während sie Ludimus testen, war der Grundgedanke. Auch war ein großer Bildschirm notwendig, wie sollen die Leute sonst das Spielgeschehen aus einer gewissen Distanz gut sehen? Zusätzlich wird noch ein lokales Netzwerk benötigt, durch welches es möglich ist sich, mit der Ludimus-App am Smartphone installiert, mit dem Laptop, welcher die Ludimus-App ebenfalls gestartet hat, zu verbinden. Noch ein paar Deko-Elemente und schon ist der Ludimus-Stand für das AEC Festival fertig. Fehlte nur noch ein Spiel welches schnell und lustig war aber trotzdem noch so anspruchsvoll, dass es für kurze Zeit fesselt. Die Lösung: ein Jump and Run in welchem die Spieler das Ziel erreichen müssen und dabei schneller als ihre Mitspieler sein sollten. Und zur Belohnung erhält der erste im Ziel mehr Punkte und darf das Level, in welchem gespielt wird, verändern. Mit Veränderung ist gemeint, dass der Spieler ein Objekt in der Welt platzieren darf um das Level leichter oder schwerer zu machen. Und welcher Spieler zuerst dreimal gewinnt, gewinnt insgesamt. Ein kurzes aber trotzdem anspruchsvolles Spiel welches spaßig ist. Dabei gab es jedoch ein Problem, das Spiel hatte nichts mit dem Error-Thema des Festivals zu tun. Also bauten wir einen “Bug” in das Spiel ein, bei welchem die Spieler ihr Smartphone schütteln müssen um seinen Charakter wieder bewegen zu können. Dieser Bug tritt bei dem Spieler auf welcher in der jetzigen Runde gerade die meisten Punkte hat. Und fertig war es die Idee für perfekte Spiel für das Festival welches sogar zum Thema passte. 
\subsection{Realität}
Zunächst benötigten wir eine Couch. Zum Glück hatte eine Klasse unserer Schule eine Couch welche wir uns borgen durften, also wäre das wichtigste schon mal geschafft. Wegen dem Bildschirm mussten wir uns auch keine Sorgen machen denn auch hier war uns die HTL Leonding eine riesige Hilfe da sie uns einen großen Bildschirm der sonst für Präsentationen gedacht war bereit stellte. Dieser hatte zwar eine kleine Anzeigeverzögerung zwischen Laptop und Bildschirm war sonst aber perfekt. Als nächstes benötigen wir nur noch ein lokales Netzwerk, damit sich die Spieler verbinden können. Dafür wurde ein Hotspot auf dem Smartphone eines Teammitgliedes erstellt und schon konnte man spielen. Als Deko-Elemente wurden ein paar Sessel und Blumen noch hinzu gestellt und der Ludimus-Stand war offiziell komplett. Fehlte nur noch das Spiel. Dieses wurde liebevoll Jumpy genannt und ist im Kapitel “Jumpy” genauer ausgeführt.
\section{Ergebnis}
Das AEC-Festival war für Team Ludimus durchaus erfolgreich. Wir erhielten durchgehend positives Feedback und waren positiv Überrascht das es so gut läuft. Der Stand war nur sehr selten wirklich leer im Verlauf der vier Tage und wir erhielten viele Meinungen und konstruktive Kritik. Auch waren viele andere Programmierer auf dem AEC-Festival mit welchem wir uns auch austauschen konnten und uns weiterbilden konnten. Über den Verlauf des gesamten Festivals loggten sich über 150 Leute, mit ihrem Google-Account, auf der Ludimus-Website ein um die App zu downloaden. Dies war ein riesiger Erfolg für unser Projekt. Jedoch realisierten wir auch wie hartnäckig Eltern eigentlich sind wenn es sich um Videospiele oder moderne Technologie handelt. Den Satz: “Ich kann sowas ja gar nicht, muss ich gar nicht erst probieren” hörten wir locker über 20 mal Tag, auch wenn wir erklärten, dass es auch andere Spiele gibt wie zum Beispiel Poker blieben sie bei der Meinung, dass sie sowieso nicht mit dem Smartphone umgehen können und es lieber nicht probieren wollen. Dank dieses Feedback realisierten wir jedoch auch großes Problem, denn wir mussten die Eltern in Zukunft auch dazu bringen mit den Kindern zu spielen. Schließlich war das der Zweck von Ludimus. Als Lösung wurden einfach genaue Anleitung hinzugefügt und es wurde immer wieder betont wie wichtig es nicht sei, dass die Eltern mit ihren Kindern spielten.
\section{Probleme}
Eigentlich verliefen die vier Tage sehr gut, hin und wieder erhielten wir genervte Bemerkungen von Eltern die nicht mochten das ihre Kinder Videospiele spielten. Außerdem war manchen Eltern nicht bewusst, dass wir ein Projekt und nicht die Kinderabgabe sind. Aber all das bei Seite war es ein großer Erfolg für Ludimus und wir konnten viel aus dem Feedback lernen und dank der Menschen auf dem AEC-Festival unsere App noch besser machen.
