\chapter{Zusammenfassung}
Ludimus war von Beginn an als Produkt für Kunden konzipiert und wurde auch so während der Entwicklung behandelt. Das Projekt entstand unter anderem Namen während der „Think your Product“ Woche, einer neu gegründeten Schulinitiative der Htl Leonding, zur Förderung von Firmengründungen. Neben einigen Vorträgen von Business Angels und erfolgreichen Gründern, gipfelte die Schulveranstaltung in einer Projektwoche in der letzten Schulwoche vor den Semesterferien. Während dieser Zeit waren alle Teilnehmer vom regulären Unterricht freigestellt und wurden von unterschiedlichen Vertrauten der Branche gecoached und in ihrer Idee bestärkt. Das Feedback konnte gleich umgesetzt und implementiert werden. Am Ende der Woche durften alle Teams ihre Idee vor einer Jury in der Factory300, in der alten Tabakfabrik, präsentieren und erhielten noch einmal Feedback und anschließend eine einjährige Mitgliedschaft für die Factory. Die Projekte wurden bewertet und gereiht und Ludimus, damals noch ConnecTable wurde mit dem ersten Platz prämiert. Motiviert meldeten uns wir sofort für den Edison Preis an, welcher sogar Preisgeld von bis zu 3000 Euro für die ersten drei Plätze versprach. Neben Pitch Trainings und Vorträgen über die Business Plan und Finanzplan Erstellung, durften wir Kontakte mit anderen Startups knüpfen, ehe wir vor einer Jury pitchen durften. Obwohl wir bei der Preisverleihung leer ausgingen, erfuhren wir von einer Jurorin, dass wir nur knapp vierter wurden und es nächstes Jahr wieder versuchen sollten. Neben einer Teilnahme bei Startup Live, bereiteten wir uns für das nächste große Event, das AEC Festival und unseren geplanten Betastart vor. Die Erfahrungen und Vorbereitungen sind in Kapitel \ref{sec:aecf} näher beschrieben. Diese Events belehrten uns in vielen Dingen bezüglich der Unternehmensgründung, jedoch litt die Entwicklung der Spiele unter dem enormen Zeitdruck, der durch die Verteilung der Events entstand. Uns blieb meist nur ein Monat um ein vorzeigbares Spiel zu entwickeln. Erst nach dem AEC Festival konnten wir uns wieder auf die Entwicklung von Ludimus stürzen, mussten jedoch zuerst die Plattform komplett umstrukturieren, da auch diese unter Zeitdruck entstand und dringendst überarbeitet werden musste. Durch diese Verzögerungen konnten wir die meisten unserer Ziele bezüglich Spieleanzahl nicht erreichen. Das Team hinter Ludimus umfasste zur Zeit des AEC Festivals vier Personen, jedoch verließen uns diese kurz daraufhin und Ludimus als Projekt wurde bis auf unbestimmte Zeit eingefroren. Noch ist unklar ob oder wie wir das Projekt weiterentwickeln, da sich unsere Ziele und Interessen während der Entwicklung stark veränderten.
